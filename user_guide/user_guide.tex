\documentclass{article}

\title{Calculator -- User guide}
\date{}

\begin{document}
	\maketitle
	
	\tableofcontents
	
	\newpage
	
\section{Introduction}
	This document is a user manual for our IVS project calculator.\\
	The purpose of this document is to provide the user with information on how to install and run the software as well as provide information about it's use and functionality.\\
	The calculator is designed for the Ubuntu 64bit operating system.
	\newpage
	
\section{Installation}
	In order to install the calculator, firstly it is necessary to download the debpackage from our github repository at $https://github.com/Jakub-Miko/IVS\_Kalkulacka.git$. Once there in the releases section you will find the downloadable debpackages. Once you've chosen a release version you wish to use and downloaded the debpackage for it, all that's left is to run the following shell command: $apt\ install\ [DEBPACKAGE\_NAME]$.\\
	In summary:\\
	$\bullet$Visit $https://github.com/Jakub-Miko/IVS\_Kalkulacka.git$\\
	$\bullet$From the Releases section download desired package\\
	$\bullet$Once downloaded, run $apt\ install\ [DEBPACKAGE\_NAME]$
	\newpage
	
\section{Use}
	The calculator application is split into several segments:\\
	$\bullet$Input field; here your inputs are displayed.\\
	$\bullet$Accumulator; in this section the results of your previous operations are stored.\\
	$\bullet$Numeric keyboard; by clicking on the buttons in this section you store the value of the button into the input field.\\
	$\bullet$Constants and functions block; in this block the buttons are shared by the inputs for constants and functions; you can swap between them by pressing the mod button at the top.\\
	$\bullet$Settings menu; this is stored at the top of the window and let's you change the sound effect settings.\\
	\\
	In the calculator binary operations function as follows:\\
	The first input is expected to be a number, followed by the operation symbol and then a second number.\\
	The syntax for unary operations always expects the numerical input first followed by the operation.\\
	In both cases upon entering another operation symbol the result of the previous equation is calculated, stored in the accumulator and then used as the first operand for the new operation.\\
	\\
	Note:\\
	The Root operation works in the same way as any other binary function, where the first operand is the exponent and the second operand is the coefficient. 
	
\end{document}